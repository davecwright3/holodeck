% =================================================================================================
% Holodeck Methods Paper - Manuscript
% -----------------------------------
%
%
%
% =================================================================================================

% =================================================================================================
% ====   Header & Setup
% =================================================================================================
\documentclass[useAMS, usenatbib]{mnras}

% general package imports and setup
\usepackage{color}
% \usepackage{booktabs}   % Fancier tables
\usepackage{tabularx}
\usepackage{float}
% \usepackage{totcount}
% \usepackage{cuted}
% adds colors to colorspace (https://en.wikibooks.org/wiki/LaTeX/Colors)
\usepackage[dvipsnames, table]{xcolor}
\usepackage{graphicx}
% Force figures to stay in their sections (tex.stackexchange.com/q/279/22806)
\usepackage[section]{placeins}
% \usepackage{subfig}
\usepackage{multirow}
\usepackage{multicol}
% \usepackage{widetext}
% \usepackage{upgreek}    % allow alternative greek letters like $\uptau$
\usepackage{enumitem}   % customize enumerate symbols
\setlist[enumerate]{leftmargin=*}   % Set left-margin of enumerate lists to match the edge
\setlist[itemize]{leftmargin=*}   % Set left-margin of enumerate lists to match the edge
\usepackage[T1]{fontenc}

\usepackage{amsmath}
\usepackage{amssymb}

% `txfonts` : Changes math-greek fonts (I like 'sigma'), NOTE: this must be after ams imports
\usepackage{txfonts}

% easier/convenient derivatives and partial derivatives, for example diffp for partials
\usepackage[thinc]{esdiff}

\usepackage{xfrac}

\usepackage{microtype}   % fix overfull/underfull hbox issues

% For ORCID iDs
% \usepackage{tikz,xcolor,hyperref}
% \definecolor{lime}{HTML}{A6CE39}
% \DeclareRobustCommand{\orcidicon}{
% 	\begin{tikzpicture}
% 	\draw[lime, fill=lime] (0,0)
% 	circle [radius=0.16]
% 	node[white] {{\fontfamily{qag}\selectfont \tiny ID}};
% 	\draw[white, fill=white] (-0.0625,0.095)
% 	circle [radius=0.007];
% 	\end{tikzpicture}
% 	\hspace{-2mm}
% }

% \foreach \x in {A, ..., Z}{\expandafter\xdef\csname orcid\x\endcsname{\noexpand\href{https://orcid.org/\csname orcidauthor\x\endcsname}
% 			{\noexpand\orcidicon}}
% }


\graphicspath{ {figs/} }

% shift around the text on the page so looks good on Letter paper
% \voffset-.6in
% \hoffset0.2in

% definitions of custom commands
\usepackage{xifthen}

\newcommand{\tr}[1]{\textrm{#1}}
\newcommand{\trt}[1]{\textrm{\tiny{#1}}}
\newcommand{\trf}[1]{\textrm{\footnotesize{#1}}}

\newcommand{\msol}{\tr{M}_{\odot}}
\newcommand{\rsol}{\tr{R}_{\odot}}

\newcommand{\mbh}{M_\bullet}
\newcommand{\rbh}{R_\bullet}

\newcommand{\mstar}{M_\star}
\newcommand{\rstar}{R_\star}

\newcommand{\hmn}{h_{\mu \nu}}
\newcommand{\gmn}{g_{\mu \nu}}
\newcommand{\emn}{\eta_{\mu \nu}}



% \newcommand{\bigt}{\scalebox{1.2}{\ensuremath{\uptau}}}
\newcommand{\bigt}{\ensuremath{\uptau}}

% General Math Stuff
\newcommand{\E}[1]{\times\nobreak10^{#1}}
\newcommand{\ls}{\lesssim}
\newcommand{\gs}{\gtrsim}
\newcommand{\logten}[1]{\log_{10}\!\lr{#1}}

% left-right parentheses
% \newcommand{\lr}[1]{{\left(#1\right)}}
\newcommand{\lr}[2][]{
    \ifthenelse{\equal{#1}{}}{
        % omitted
        {\left(#2\right)}
    }{
        % given
        {\left(#2\right)}^{#1}
    }
}

% left-right square-bracket
% \newcommand{\lrs}[1]{{\left[#1\right]}}
\newcommand{\lrs}[2][]{
    \ifthenelse{\equal{#1}{}}{
        % omitted
        {\left[#2\right]}
    }{
        % given
        {\left[#2\right]}^{#1}
    }
}


% left-right-tight (parenthesis
\newcommand{\lrt}[1]{{\left(\!#1\!\right)}}

% left-right-tight
\newcommand{\lrst}[1]{{\left[\!#1\!\right]}}

% optional first argument for exponent
%    i.e. `\scale{A}{B} = (A/B)` or `\scale[2]{A}{B} = (A/B)^2`
% \newcommand{\scale}[3][]{\lr{ \frac{#2}{#3} }^{#1}}
% ----------
% \newcommand{\scale}[3][]{
%     \ifthenelse{\equal{#1}{}}{
%         % omitted
%         \lr{ \! \frac{#2}{#3} \!}
%     }{
%         % given
%         {\lr{ \! \frac{#2}{#3} \!}\!}^{#1}
%     }
% }
% -----------
\newcommand{\scale}[3][]{
    \ifthenelse{\equal{#1}{}}{
        % omitted
        \lr{ \frac{#2}{#3} }
    }{
        % given
        {\lr[#1]{ \frac{#2}{#3} }}
    }
}

\newcommand{\sinpar}[2][]{\sin^{#1}\!\lr{#2}}
\newcommand{\cospar}[2][]{\cos^{#1}\!\lr{#2}}
\newcommand{\blank}{\, ... \,}

\newcommand{\lt}{<}
\newcommand{\gt}{>}

\newcommand{\volfactor}{\Lambda_{ij}}
\newcommand{\poisson}{\mathcal{P}}

\newcommand{\parkes}{Parkes}
\newcommand{\nanograv}{NANOGrav}
\newcommand{\nanogravplus}{NANOGrav+}
\newcommand{\iprime}{IPTA$'$}
\newcommand{\iptazero}{IPTA$_0$}
\newcommand{\iprimeone}{IPTA$'_1$}
\newcommand{\iprimerap}{IPTA$'_\mathrm{rap}$}

\newcommand{\aul}{A_\trt{yr$^{-1}$,ul}}
\newcommand{\hc}{h_\trt{c}}
\newcommand{\hs}{h_\trt{s}}
\newcommand{\hsc}{h_\trt{s,circ}}
\newcommand{\hscirc}{h_\trt{s,circ}}
\newcommand{\hcs}{h_\trt{c,s}}

% \newcommand{\radius}{r}

\newcommand{\mdot}{\dot{M}}
\newcommand{\mdotedd}{\dot{M}_\trt{Edd}}
\newcommand{\mdotill}{\dot{M}_\trt{ill}}
\newcommand{\ledd}{L_\trt{Edd}}    % Eddington luminosity
\newcommand{\lacc}{L_\trt{acc}}    % accretion luminosity
\newcommand{\radeff}{\varepsilon_\trt{rad}}

% \newcommand{\powerlaw}{PPL}
\newcommand{\scalenoise}{\lambda_\tr{noise}}

% == Logistical Commands ==
\newcommand{\secref}[1]{\textsection\ref{#1}}
\newcommand{\chref}[1]{Ch.~\ref{#1}}          % Reference a chapter
\newcommand{\figref}[1]{Fig.~\ref{#1}}
\newcommand{\refeq}[1]{{Eq.~\ref{#1}}}
\newcommand{\tabref}[1]{{Table~\ref{#1}}}
\newcommand{\needcite}{{\color{red}[???]}}
\newcommand{\missing}{ {\color{red}??|??|??} }
\newcommand{\result}[1]{ {\color{blue}#1} }
\newcommand{\fiducial}[1]{ {\color{ForestGreen}#1} }
\newcommand{\NOTE}[1]{\noindent\textbf{\color{red}!!#1!!}}
\newcommand{\note}[1]{{\color{violet}#1}}
\newcommand{\fnm}[1]{\footnotemark[#1]}
\newcommand{\fnt}[2]{\footnotemark[#1]{#2}}
\newcommand{\mc}[2]{\multicolumn{#1}{c}{#2}}
\newcommand{\mr}[2]{\multirow{#1}{*}{#2}}
\newcommand{\ccba}{\cellcolor{blue!20}}
\newcommand{\ccbb}{\cellcolor{blue!30}}
\newcommand{\ccbc}{\cellcolor{blue!40}}

% == Terminology Shortcuts ==
\newcommand{\mbhb}{MBHB}  % {MBH-Binaries}
\newcommand{\heavy}{\textit{heavy}}  % 'Heavy' Binaries
\newcommand{\major}{\textit{major}}  % 'Major-Mergers' Binaries
\newcommand{\heavymajor}{\heavy~\&~\major}
\newcommand{\heavydef}{{\textit{heavy} ($M > 10^{9} \, \msol$)}}  % 'Heavy' Binaries
\newcommand{\majordef}{{\textit{major} ($\mu > 0.1$)}}  % 'Major-Mergers'

% == Math Symbols ==
\newcommand{\bmax}{b_\trt{max}}   % maximum impact parameter
\newcommand{\vm}{v_\trt{M}}
\newcommand{\vs}{v_\star}
\newcommand{\vsound}{c_s}   % sound speed, speed of sound
\newcommand{\hrad}{r_\trt{H}}      % hardening radius
\newcommand{\dmdens}{\rho_\trt{DM}}
\newcommand{\nfwrad}{r_\trt{NFW}}
\newcommand{\nfwc}{c_\trt{NFW}}
\newcommand{\mvir}{M_\trt{v}}
\newcommand{\mrat}{\mu}    % Mass-Ratio
\newcommand{\rvir}{R_\trt{v}}
\newcommand{\rlc}{\mathcal{R}_{lc}}
\newcommand{\rinfl}{\mathcal{R}_\trt{infl}}   % Radius of influence (for MBH)
\newcommand{\risco}{\mathcal{R}_\trt{isco}}
\newcommand{\rs}{R_\trt{s}}    % Stellar radius
\newcommand{\rb}{R_\trt{b}}    % Bound radius
\newcommand{\rh}{R_\trt{h}}    % Hard radius
\newcommand{\mpro}{m_\trt{p}}    % Proton Mass
\newcommand{\sigmat}{\sigma_\trt{T}}   % Thomson-Scattering Cross-Section
% \newcommand{\trel}{\tau_\trt{rel}}
% \newcommand{\torb}{\tau_\trt{orb}}
% \newcommand{\tcross}{\tau_\trt{cross}}
\newcommand{\trel}{\bigt_\trt{rel}}
\newcommand{\torb}{\bigt_\trt{orb}}
\newcommand{\tcross}{\bigt_\trt{cross}}
\newcommand{\mce}{\mathcal{E}}
\newcommand{\rhoc}{\rho_\trt{c}}
\newcommand{\nfwdens}{\rho_{\trt{DM},0}}

\newcommand{\pgw}{P_\trt{GW}}    % GW Period
\newcommand{\pgwmax}{P_\trt{GW,max}}    % GW Period
\newcommand{\tobs}{T_{\!\trt{obs}}}   % Observing Duration (Pulsar Time)
\newcommand{\dtobs}{\Delta t_\trt{obs}}   % Observing Cadence / Period (Pulsar Time)


\newcommand{\ms}{m_\star}
\newcommand{\rstarhalfmass}{R_{\star,\trt{1/2}}}
\newcommand{\omgw}{\Omega_\trt{GW}}   % ratio of GW energy density to critical density
\newcommand{\rhogw}{\rho_\trt{GW}}    % density of GW
\newcommand{\pyr}{\textrm{yr}^{-1}}
\newcommand{\ayr}{A_{\trt{yr}^{-1}}}        % GWB amplitude normalization at 1/yr
\newcommand{\atyr}{A_{{\scriptscriptstyle0.1} \trt{yr}^{-1}}}  % GWB amplitude normalization at 0.1/yr

\newcommand{\hubdist}{D_\trt{H}}   % 'Hubble Distance' c/H_0
\newcommand{\comdist}{d_\trt{c}}   % 'Comoving Distance'
\newcommand{\lumdist}{d_\trt{L}}   % 'Comoving Distance'
\newcommand{\astropy}{\texttt{Astropy}}
\newcommand{\matplotlib}{\texttt{matplotlib}}
\newcommand{\numpy}{\texttt{NumPy}}
\newcommand{\scipy}{\texttt{SciPy}}
\newcommand{\ipython}{\texttt{ipython}}
\newcommand{\jupyter}{\texttt{jupyter}}
\newcommand{\arepo}{{\texttt{Arepo}}}
\newcommand{\sympy}{{\texttt{SymPy}}}
% \newcommand{\kalepy}{{\texttt{kalepy}}}

% \DeclareRobustCommand\kalepy{\texttt{
%     % \scalebox{1.1}{%
%     {\fontsize{10}{8} \selectfont
%     k\kern-0em%
%     % \raisebox{0ex}{\textcolor{blue}{$a$}}\kern-.14em%
%     \raisebox{0ex}{$a$}\kern-.14em%
%     \textcolor{black}{$l$}\kern-0.05em%
%     e\kern+0.04em%
%     }%
%     p\kern-.05em%
%     y%
% }}

\DeclareRobustCommand\kalepy{\texttt{
    % \scalebox{1.1}{%
    % {\fontsize{10}{8} \selectfont
    k\kern+0.05em%
    % \raisebox{0ex}{\textcolor{blue}{$a$}}\kern-.14em%
    \raisebox{0ex}{$a$}\kern-.12em%
    \textcolor{black}{$l$}\kern-0.0em%
    e\kern+0.04em%
    % }%
    p\kern-.05em%
    y%
}}



\newcommand{\pc}{\mathrm{pc}}
\newcommand{\yr}{\mathrm{yr}}        % yr in math-mode

\newcommand*\ave[1]{\overline{#1}}
\newcommand\erfc[1]{\mathrm{erfc}\left(#1\right)}
\newcommand\erfcinv[1]{\mathrm{erfc}^{-1}\left(#1\right)}

% \newcommand{\SNR}{\tr{S}/\tr{N}}   % Signal-to-Noise Ratio SNR
% \newcommand\snra{\mbox{$\mathcal{R}_A$}}
% \newcommand\snrb{\mbox{$\mathcal{R}_B$}}
% \newcommand\snrta{\mbox{$\mathcal{R}_A^T$}}
% \newcommand\snrtb{\mbox{$\mathcal{R}_B^T$}}
% \newcommand\pulsarsum{\sum_k \sum_{ij}}
% \newcommand\Gij{\Gamma_{ij}}

\newcommand{\rdf}{R_\trt{df}}
\newcommand{\tdf}{\bigt_\trt{df}}
\newcommand{\mtot}{M_\trt{tot}}

\newcommand{\fcoal}{\mathcal{F}_\trt{coal}}
\newcommand{\fstall}{\mathcal{F}_\trt{stall}}
\newcommand{\frefill}{\mathcal{F}_\trt{refill}}
\newcommand{\flcsix}{$\frefill = 0.6$}
\newcommand{\flcten}{$\frefill = 1.0$}
\newcommand{\lmdot}{\lambda_{\dot{M}}}   % pre-eddington mdot scaling
\newcommand{\fedd}{f_\trt{Edd}}    % Mdot-eddington factor for limiting
\newcommand{\rsg}{\mathcal{R}_\trt{SG}}
\newcommand{\rsgmax}{\mathcal{R}_\trt{SG,Max}}
\newcommand{\dfatten}{f_\trt{DF,LC}}
\newcommand{\mchirp}{\mathcal{M}}     % Chirp-mass
\newcommand{\mseed}{M_\trt{seed}}    % seed-mass BH
\newcommand{\diffco}{\mathcal{D}_{v^2}}   % Diffusion coefficient
\newcommand{\cosmocorr}{\Gamma_\trt{cos}}   % 'Cosmological Correction' factor
\newcommand{\illlength}{L_\trt{ill}}  % 'Illustris Length' box-length
\newcommand{\vollc}{V_\trt{c,lc}}  % comoving volume of the past light-cone
\newcommand{\volcom}{V_\trt{c}}   % comoving volume
\newcommand{\volill}{V_\trt{ill}}   % Illustris volume
\newcommand{\illcosmo}{$H_0 = 70.4 \textrm{ km s}^{-1} \textrm{ Mpc}^{-1} \, (h = 0.704), \,
\Omega_m = 0.2726, \, \Omega_\Lambda = 0.7274$}
\newcommand{\tlb}{t_\trt{LB}}
\newcommand{\thubble}{\bigt_\tr{Hubble}}
\newcommand{\tfhard}{\bigt_\tr{h}^{f}}
\newcommand{\thard}{\bigt_\tr{h}}
\newcommand{\thardi}{\bigt_{\tr{h},i}}
\newcommand{\thardij}{\bigt_{\tr{h},ij}}
\newcommand{\thardgw}{\bigt_\tr{gw}}
\newcommand{\thardvd}{\bigt_\tr{vd,1}}
\newcommand{\tdyn}{\bigt_\trt{dyn}}
\newcommand{\tgw}{\bigt_\tr{gw}}
\newcommand{\tgwi}{\bigt_{\trt{gw},i}}
\newcommand{\tvisc}{\bigt_v}
\newcommand{\ttherm}{\bigt_\trt{therm}}
\newcommand{\tviscone}{\bigt_\tr{v,1}}
\newcommand{\tvisctwo}{\bigt_\tr{v,2}}
\newcommand{\lgw}{L_\tr{GW}}
\newcommand{\lgwc}{L_\tr{GW,circ}}
\newcommand{\rgap}{\lambda_\trt{gap}}
\newcommand{\rcrit}{\mathcal{R}_\trt{crit}}
\newcommand{\rselfgrav}{\lambda_\trt{sg}}
\newcommand{\fgw}{f_\tr{GW}}
\newcommand{\fgwc}{f_\tr{GW,circ}}
\newcommand{\qdisk}{q_B}
\newcommand{\reg}[1]{\textit{Region-#1}}
\newcommand{\egw}{\varepsilon_\trt{GW}}   % energy in GW
\newcommand{\fluxlc}{F_\trt{lc}}
\newcommand{\fluxflc}{F^\trt{full}_\trt{lc}}
\newcommand{\fluxsslc}{F^\trt{eq}_\trt{lc}}
\newcommand{\msp}{\,\,\,\,}
\newcommand{\wnrms}{\sigma_\trt{WN}}   % White-Noise Sigma RMS
\newcommand{\nrms}{\sigma_\trt{N}}
\newcommand{\micros}{\mu \textrm{s}}
\newcommand{\nanos}{\textrm{ns}}
\newcommand{\rnamp}{A_\trt{RN}}
\newcommand{\fyr}{{f_\trt{yr}}}
\newcommand{\freqper}[1]{1/\left(#1 \, \yr\right)^{-1}}
\newcommand{\scell}[1]{\begin{tabular}{@{}c@{}}#1\end{tabular}}
\newcommand{\rnind}{{\protect\scalebox{1.1}{\ensuremath{\gamma{}}}_\trt{RN}}}
\newcommand{\forefac}{\lambda_\textrm{fore}}
\newcommand{\hcfore}{h_c^\textrm{fore}}
\newcommand{\hcback}{h_c^\textrm{back}}
\newcommand{\eccinit}{e_0}

\newcommand{\tvar}{\bigt_\trt{var}}
\newcommand{\zmax}{z_\trt{max}}
\newcommand{\qcrit}{q_\trt{crit}}
\newcommand{\qmin}{q_\trt{min}}
\newcommand{\dl}{d_\trt{L}}   % luminosity distance
\newcommand{\dc}{d_c}   % comoving distance

\newcommand{\massfunc}{\psi_M}
\newcommand{\mergrate}{\mathcal{R}_{gg}}
\newcommand{\ndens}{\phi}
\newcommand{\mdotrat}{\lambda}
% \newcommand{\dfdop}{\delta F_\nu^\mathrm{d}}
% \newcommand{\dfhyd}{\delta F_\nu^\mathrm{h}}
% \newcommand{\dfsens}{\delta F_\trt{sens}}
% \newcommand{\dffloor}{\delta F_\trt{floor}}
% \newcommand{\fsens}{F_{\nu,\trt{sens}}}
% \newcommand{\fsenslsst}{F_{\nu,\trt{sens}}^\trt{LSST}}
% \newcommand{\fsenscrts}{F_{\nu,\trt{sens}}^\trt{CRTS}}

\newcommand{\old}[1]{{\color{gray}#1}}
\newcommand{\sfluxunits}{\textrm{ erg/s/Hz/cm}^2}
\newcommand{\invmpccubed}{\textrm{Mpc}^{-3}}
\newcommand{\invlogt}{\left(\log \, \torb\right)^{-1}}
\newcommand{\invagn}{\textrm{AGN}^{-1}}
\newcommand{\invarcsecsq}{\textrm{arcsec}^{-2}}
% \newcommand{\arcsec}{\textrm{as}}
\newcommand{\as}{\textrm{arcsec}}

\newcommand{\fracobs}{f_\trt{obs}}
\newcommand{\fracobsi}{f_{\trt{obs},i}}
\newcommand{\feddsys}{f_\trt{Edd,sys}}
\newcommand{\feddi}{f_{\trt{Edd},i}}
\newcommand{\feddtwo}{f_{\trt{Edd},2}}

% \defcitealias{BBR80}{Begelman1980}
% \defcitealias{paper1}{KBH-16}

\defcitealias{hkm09}{HKM09}
\defcitealias{Begelman1980}{BBR80}
\defcitealias{Rosado1503}{RSG15}
\defcitealias{paper1}{Paper-1}
\defcitealias{paper2}{Paper-2}
\defcitealias{paper3}{Paper-3}
\defcitealias{paper4}{Paper-4}

% journal abbreviations for bibliography

% \def\apj{ApJ}
% \def\mnras{MNRAS}
% \def\nat{Nat}
% \def\physrevB{Phys. Rev. B}
% \def\prd{Phys. Rev. D}
% \def\araa{ARA\&A}                % "Ann. Rev. Astron. Astrophys."
% \def\aap{A\&A}                   % "Astron. Astrophys."
% \def\aaps{A\&AS}                 % "Astron. Astrophys. Suppl. Ser."
% \def\aj{AJ}                      % "Astron. J."
% \def\apjs{ApJS}                  % "Astrophys. J. Suppl. Ser."
% \def\pasp{PASP}                  % "Publ. Astron. Soc. Pac."
% \def\apjl{ApJ}                   % letter at ApJ
% \def\pasj{PASJ}
% \def\ssr{Space Science Reviews}
% \def\physrep{Physics Reports}
% \def\qjras{Quarterly Journal of the Royal Astronomical Society}

\def\lt{<}

\def\aapr{A\&AR}                                     % Astronomy and Astrophysics Review, the
\def\aj{Astron. J.}              		   		% Astronomical Journal
\def\apj{Astrophys. J.}       		        	 	% Astrophysical Journal
\def\apjl{Astrophys. J. Lett.}             		% Astrophysical Journal, Letters
\def\pasj{PASJ}
\def\physrep{Phys. Rep.}
\def\pasp{PASP}
\def\pasa{PASA}
\def\ssr{Space Science Rev.}			% Space Science Reviews
\def\apjs{Astrophys. J., Suppl. Ser.}            % Astrophysical Journal, Supplement
\def\mnras{Mon. Not. R. Astron. Soc.}        % Monthly Notices of the RAS
\def\prd{Phys. Rev. D}      				% Physical Review D
\def\prl{Phys. Rev. Lett.}   				% Physical Review Letters
\def\cqg{Class. Quant. Grav.}			%Classical and Quantum Gravity
\def\araa{Annu. Rev. Astron. Astrophys.}  % Annual Review of Astron and Astrophys
\def\nat{Nature}              				% Nature
\def\na{Nature Astron.}                     % Nature Astronomy
\def\aap{Astron. Astrophys.}               		% Astronomy and Astrophysics
\def\jasa{J. Am. Stat. Assoc.}
\def\jrssb{J. R. Stat. Soc. B}
\def\aipcs{AIP Conf. Ser.}
\def\jgr{J. Geophys. Res.}                      % Journal of Geophysical Research
\def\sovast{Soviet Astronomy}
\def\planss{Planet.~Space~Sci.}   % Planetary Space Science
\def\memsai{Mem. Societa Astronomica Italiana}


% =================================================================================================
% ====    Front Matter
% =================================================================================================

\newtotcounter{citnum} % From the package documentation
\def\oldbibitem{} \let\oldbibitem=\bibitem
\def\bibitem{\stepcounter{citnum}\oldbibitem}

\newcommand{\orcidauthorA}{0000-0002-6625-6450} % For author A

\title[holodeck: MBH binary populations]{Enter the \texttt{holodeck}: massive black hole binary population synthesis for gravitational wave calculations}
\author[NANOGrav]{the NANOGrav collaboration}



\begin{document}

\maketitle

\begin{abstract}
    abstract
\end{abstract}

\begin{keywords}
    quasars: supermassive black holes
\end{keywords}




% =================================================================================================
% ====    Section 1 - Introduction
% =================================================================================================

\begin{itemize}
    \item Change $f_r \rightarrow f$
    \item holodeck uses $f_\trt{obs}$ as grid variable, not $f_r$, mesh this with the analytic formalism and description
\end{itemize}


\section{Introduction}
    \label{sec:intro}

    Massive black holes (MBHs) are known to occupy the centers of most if not all massive galaxies \needcite{}.
    Binaries of MBHs are believed to form following the merger of massive galaxies \needcite{}.  Such MBH Binaries (MBHBs) could then produce low-frequency gravitational waves (GWs) \needcite{LF-GWs}, in the nano-Hertz regime, that would soon be detectable by experiments called pulsar timing arrays (PTAs) \needcite{PTAs}.  In this paper, we introduce the MBHB population synthesis code called \holodeck{}.  The goal of this package is to provide a cohesive and comprehensive framework for:
    \begin{enumerate}
        \item
        modeling the physics of MBHB formation and evolution while accurately capturing the true uncertainties in this modeling;
        \item
        rapidly generating large simulated populations of MBHBs;
        \item
        calculating the resulting GW signals from these populations.
    \end{enumerate}
    % (1) modeling the physics of MBHB formation and evolution while accurately capturing the true uncertainties in this modeling, (2) rapidly generating large simulated populations of MBHBs, and (3) calculating the resulting GW signals from these populations.

    As in the case of high-frequency GWs detected by LIGO-Virgo \needcite{}, tremendous uncertainties exist in the physics of the formation and evolution of the binary sources.  For individual MBHs, there are numerous, well-known scaling relationships between their properties and those of their host galaxies \needcite{MBH-Host relations}, but numerous questions remain such as where or not the observed sample of MBHs (i.e.~quasars and active galactic nuclei, AGN) accurately reflect the full population \needcite{MBH bias}, and how reliably these relationships extend to different host galaxies and redshift ranges.

    For binary MBHs in particular, a variety of physical processes are required for the two MBHs to find the center of the post-merger host galaxy and eventually become gravitationally bound \needcite{BBR1980}.  In particular, the two MBHs must interact with their host galaxy environments through dynamical friction, stellar scattering, accretion disk torques, and possibly three-body interactions with a third MBH component \needcite{}, before reaching small enough binary separations for GW emission to eventually dominate the final coalescence of the binary.  Besides GW emission, none of these processes can be modeled exactly, necessitating the usage of simplified prescriptions which each containing numerous parametric uncertainties.  The resolution of cosmological hydrodynamic simulations is far too coarse to resolve the MBH binary phase, with some simulations instantaneously merging nearby MBHs \needcite{illustris}, and others using sub-grid models to `resolve' down to the scale of $\sim kpc$ \needcite{Romulus/Tremmel, Ma+Hopkins}.

    In addition to the theoretical uncertainties, there exists a substantial dearth of observational constraints.  A growing number of \textit{candidate} MBHB systems have been identified through varying electromagnetic (EM) indicators in AGN \needcite{}, but a substantial fraction may be false positives \needcite{}.  To date, no gravitationally bound MBH binary has yet to be confirmed, although examples exist of dual AGN at close projected separation \needcite{Rodriguez et al. 2007}, and single AGN showing time-variability highly suggestive of a binary companion \needcite{OJ~287}.

    GWs from MBHBs, particularly in the form of a stochastic gravitational wave background (GWB), have been forecast for decades.  Only recently have the sensitivities of PTA experiments begin to encroach on astrophysically plausible GWB amplitudes \needcite{PTA limits} \needcite{c.f.~Shannon et al.}.  Indeed, PTA upper-limits have already been used to place constraints on limited sub-spaces of MBHB parameters \needcite{11yr analysis, etc}.  In the 12.5yr dataset of the North-American nano-Hertz Observatory for Gravitational waves \needcite{NANOGrav}, a `common process' signal was identified that is consistent with a GWB but not yet statistically definitive.  Followup modelling suggests that the GW nature of the signal can be determined within the next few years \needcite{Pol}.  While the interpretation is currently entirely conjectural, the two currently measurable parameters of the common process, the GW-strain amplitude and strain spectral index, each challenge the standard expectations of theoretical models.  The \todo{possible} amplitude is larger than many models have predicted, and the spectral index significantly flatter than the canonical `-2/3' power-law \needcite{phinney} \needcite{c.f.~Sesana, Kocsis}.  Still, both of these parameters can easily be fit with current models, and indeed, studies have already performed MBHB model inference using the amplitude and spectral index of the candidate \todo{(what's the word besides `candidate'?)} signal \needcite{middleton}.

    Studies which have used PTA observations to constrain MBHB models have been focused on demonstrating the feasibility of performing such inference, and thus have relied on single classes of models (semi-analytic models, SAMs) that make significant simplifications of the physical processes, and only explore small sub-spaces of astrophysical parameters.  The \holodeck{} framework, presented here, is structured to more fully capture the astrophysical uncertainties underlying our understanding of MBHB formation and evolution.  \holodeck{} uses comprehensive and self-consistent models for binary evolution, in addition to allowing SAMs to be interchanged with populations derived from cosmological hydrodynamic simulations.  In this way, \holodeck{} is designed to capture both statistical and systematic uncertainties by allowing the marginalization over not only specific astrophysical parameters, but also entire modeling methodologies.

    In this paper, we present the simulation methodologies and their implementations in \holodeck{}.  We also use the framework to explore in more detail the range of plausible GW-strain spectral indices that would be detectable by PTAs. \todo{expand this paragraph}.



% =================================================================================================
% ====    Section 2 - Methods
% =================================================================================================



\section{Methods}
    \label{sec:meth}

    \subsection{Source Distributions and Gravitational Wave Calculations}

        It was shown by \citet[][]{Phinney-2001} that the GWB spectrum can be calculated by relating the current energy-density of GWs to the integrated GW energy produced by sources over all of cosmic time,
        \begin{equation}
            \label{eq:soltan_analog}
            \hc^2(f) = \frac{4G}{\pi c^2 f} \int dz \; \diffp{\ndens}{{z}} \, \diffp{\egw}{{f_r}} \bigg|_{f_r=f\lr{1+z}}.
        \end{equation}
        Converting from the rest-frame frequency $f_r$ to observer-frame $f$ in the term in front of the integral cancels out the redshifting of gravitons between their emission and detection.  Here the comoving, volumetric number density of sources is \mbox{$\ndens \equiv \diffp{N}{{V_c}}$}, and the GWB spectrum is described in terms of the characteristic strain $\hc$.  We can convert from the GW energy-spectrum integrated over each sources lifetime, $\diffp{\egw}{{f_p}}$, to GW spectral strain using Eq.~\ref{eq:gw_energy_spectrum}~\&~\ref{eq:strain_lum}, and rewrite the GWB spectrum as,
        \begin{equation}
            \label{eq:gwb_sa}
            \hc^2(f) = \int_0^\infty \!\! dz \, \sum_n^\infty \; \frac{d\ndens}{dz} \, \hsn^2\lr{f_p} \cdot 4\pi c \, d_c^2 \cdot \lr{1+z} \cdot \thard(f_p) \, \bigg|_{f_p = f(1+z)/n}.
        \end{equation}
        In the case of purely GW-driven binary hardening, and circular binary orbits, this can be simplified to \citep[][Eq.~11]{Phinney-2001},
        \begin{equation}
            \label{eq:gwb_ideal}
            \hc^2(f) = \frac{4 \pi}{3 c^2} \, \lr[-4/3]{2\pi f} \int_0^\infty \!\! dz \; \frac{d\ndens}{dz}  \, \frac{\lr[5/3]{G \mchirp}}{\lr[1/3]{1+z}}.
        \end{equation}
        This ideal, power-law GWB is often expressed as, \mbox{$\hc(f) = A_{\yr} \cdot \lr[-2/3]{f \cdot \yr}$}.

        The redshift evolution of the comoving number density of sources ($\ndens \equiv dN/dV_c$) can be related to the number of sources in a given frequency interval as \citep[][Eq.~6]{Sesana+2008},
        \begin{equation}
            \label{eq:number_density_to_number_frequency}
            \frac{d\ndens}{dz} = \frac{d^2N}{dz \, dV_c} = \frac{d^2N}{dz \, d\ln f_p} \frac{d \ln f_p}{dt_r} \frac{dt_r}{dz} \frac{dz}{dV_c}.
        \end{equation}
        Plugging \eqref{eq:number_density_to_number_frequency} into \eqref{eq:gwb_sa}, and utilizing the expressions from Eqs.~\ref{eq:gw_hard},~\ref{eq:hard_time},~\ref{eq:comvol}~\&~\ref{eq:redshift_time}, to write:
        \begin{equation}
            \label{eq:gwb_mc}
                \hc^2(f) = \int dz \sum_n^\infty \frac{d^2 N}{dz \, d\ln f_p} \, \hsn^2(f_p).
        \end{equation}

        While Eq.~\ref{eq:gwb_sa} and Eq.~\ref{eq:gwb_mc} are formally equal, how they are used in practice can produce important differences.  In particular, as pointed out by \citep[][Eq.~6]{Sesana+2008}, `monte-carlo' calculations that explicitly use Eq.~\ref{eq:gwb_mc} encode the intrinsic discreteness of binary sources.  The `semi-analytic' calculation using Eq.~\ref{eq:gwb_sa}, naively implemented, assumes a smooth, continuous distribution of binary sources without taking into account their quantization.  The effect of binary quantization on the GWB is to produce a steeping of the GWB spectrum relative to the idealized power-law \eqref{eq:gwb_ideal}.  The frequency at which this becomes important is model dependent, but typically at \mbox{$f \gtrsim 1 - 10 \, \pyr$}.

    \subsection{Discretization and Cosmic Variance}

        The conversion from a continuous number-density of sources, to a discrete population of binaries is given by \eqref{eq:number_density_to_number_frequency}.  This can also clearly be seen by comparing Eq.~\ref{eq:gwb_mc} with Eq.~\ref{eq:gwb_sa}, to identify:
        \begin{equation}
            \label{eq:pop_discrete_from_continuous}
            \frac{d^2 N}{dz \, d\ln\!f_p} =
                \frac{d \ndens}{dz} \, 4\pi c \, d_c^2 \lr{1+z} \, \thardf(f_p).
        \end{equation}

        In general, implementations of binary populations will be described over some finite grid, often in a parameter space of total mass, mass ratio, redshift, and orbital frequency.  In this case the derivative over redshift in \eqref{eq:pop_discrete_from_continuous}, should be replaced with one over all parameters of interest, e.g.~\mbox{$d\ndens / dz \rightarrow d^3\ndens / dM \, dq \, dz$}.  All additional parameters are treated the same in this context, so for brevity we will only explicitly include redshift.

        Eventually the end goal of binary modeling is often to determine the number of binaries over some range of parameters.  When calculating gravitational wave signals, for example, the number of binaries across a target frequency interval is desired.  The discreteness of binary sources can be enforced by converting the differential value of \eqref{eq:pop_discrete_from_continuous} into a number of sources over a small bin, i.e., \mbox{$\Delta N \approx \lr{d^2 N / dz \, d\ln\!f_p} \, \cdot \Delta z \cdot \Delta \ln\!f_p$}.

        In general, the properties of \textit{every} binary in the Universe are not known, and thus the MC calculation cannot be performed directly.  The typical solution is to use some version of \eqref{eq:number_density_to_number_frequency} to convert from a number-density (e.g.~$d\ndens/dz$) into a discrete distribution of individual sources (e.g.~$d^2 N / dz \, d\ln f_r$).  The number-density is often constructed from either (semi-)analytic models which specify $\ndens(M,q,z,\dots)$ explicitly, or from simulations/observations of sources in a finite volume.  In the latter case, the number-density will be calculated as $\ndens \equiv N_\trt{finite}/V_\trt{c,finite}$, often over bins in some parameter space (e.g.~$\{M, q, z, \dots\}$). \\

        For a finite population of binaries, the GWB spectrum is only defined over finite bins of frequency.  The definition of characteristic strain is related to the GW energy across a logarithmic frequency bin, so \eqref{eq:gwb_mc} should really be written as,
        \begin{equation}
            \label{eq:gwb_discrete}
            \hc^2(\Delta \ln f) = \sum_\textrm{bins} \sum_n^\infty \frac{\Delta N(f_p,M_a,q_b,z_c,\dots)}{\Delta \ln f_p} \, \hsn^2(f_p,M_a,q_b,z_c,\dots).
        \end{equation}
        Here we have written out explicitly that the number of binaries, $\Delta N(f_p,M_a,q_b,z_c,\dots)$, is the number sources with frequencies inside $\Delta \ln f_p$, typically over bins of mass, mass-ratio, redshift, etc.  Discretizing and plugging into Eq.~\ref{eq:number_density_to_number_frequency} (or by examining Eq.~\ref{eq:gwb_sa}), we can then write,
        \begin{equation}
            \label{eq:expectation_number_of_binaries}
            \langle \Delta N \rangle = \Delta \ndens \cdot 4\pi c \, \distcom^2 \cdot (1+z) \cdot \thard \cdot \Delta \ln f_p.
        \end{equation}
        Note that we have again dropped the written dependence on bin values $(M_a, q_b, \dots)$, but they are still implicitly present.  This expression represents an \textit{expectation value} based on extrapolating the given number-density of binaries.  It also will not give an integer, discrete number of binaries.  To address this, we construct discrete realizations by drawing from a Poisson distribution: $\Delta N \in \poisson(\langle \Delta N \rangle)$. \\

        The way \eqref{eq:expectation_number_of_binaries} is handled in practice will still vary somewhat depending on how $\ndens$ is specified.  If a semi-analytic model is being used, then often
        $\Delta \ndens = \left[ d^3 \ndens(M,q,z) / dM \, dq \, dz \right] \cdot \Delta M \, \Delta q \, \Delta z$.  In the case of a finite volume population, the procedure can be simplified.  An analogous exercise would amount to a binning and un-binning of the sources, i.e.~$\Delta \ndens \big|_\textrm{finite} = \left[ \Delta N_\textrm{finite}(M,q,z) / \Delta M \, \Delta q \, \Delta z \, \Delta V_c \right] \cdot \Delta M \, \Delta q \, \Delta z$, which unnecessarily reduces accuracy.  Instead, we can imagine shrinking each of our parameter-space bins such that each one captures a single source, in which case we can write,
        \begin{equation}
            \label{eq:expectation_number_of_single_binary}
            \langle \Delta N \rangle_\textrm{source} = \frac{4\pi c \, \distcom^2 \, \thard}{V_c} \cdot (1+z) \cdot \Delta \ln f_p.
        \end{equation}
        In this case, $\thard$ and $\hsn$ in \eqref{eq:gwb_discrete}, are the hardening time and strain amplitude for each source, and $\langle \Delta N \rangle_\textrm{source}$ represents the expectation number of binaries in the Universe implied by each source in the finite volume.

    % \subsection{??Other??}

    %     Let $N(M,q,z,\frst)$ denote the number of binaries up to a total mass $M$, mass ratio $q$, redshift $z$, and rest-frame orbital-frequency $\frst$.
    %     We can calculate the characteristic strain from an ensemble of GW sources as \citep[][Eqs.~5/8]{Phinney-2001},
    %         \begin{align}
    %             h_c^2 = \int_0^\infty \!\! dz \; \frac{d^4 N}{dM \, dq \, dz \, d\ln\!\frst} \; \hs^2\lr{\frst}.
    %         \end{align}
    %     In practice, the total number of sources in the Universe is calculated from the number-density, $n \equiv dN / dV_c$, for some comoving volume $V_c$, extrapolated into an observer's entire past light-cone.  These can be related as \citep[][Eq.~6]{Sesana+2008},
    %         \begin{align}
    %             \label{eq:num_num_dens}
    %             \frac{d^4 N}{dM \, dq \, dz \, d\ln\!\frst} = \frac{d^3 n_c}{dM \, dq \, dz} \frac{dz}{dt} \frac{dt}{d\ln\!\frst} \frac{d V_c}{dz}.
    %         \end{align}
    %     Note that the total number of binaries in a given frequency interval depends not only on the number density, but also their rate of frequency-evolution.  The faster binaries evolve through a given frequency interval, the fewer binaries will occupy that interval on average.  This is typically described by the binary `hardening' timescale (or `residence time'), given by the expression,
    %     \begin{equation}
    %         \label{eq:hardening}
    %         \thardf \equiv - \diffp{t}{{\ln\!\frst}} = - \frac{\frst}{\partial \frst / \partial t}.
    %     \end{equation}

    %     % Plugging in the relevant cosmographic relationships (see \secref{sec:app_eqs}) we have,
    %     %     \begin{align}
    %         %         h_c^2(f) & = \int_0^\infty \!\! dz \; \frac{dn_c}{dz} \, h^2\lr{f_r} \, 4\pi c \, d_c^2 \lr{1+z} \, \thardf.
    %         %     \end{align}
    %         % The challenge in modeling astrophysical sources of GWs is in calculating their number density.  Numerous approaches are possible, described in the following sections.

    %     Plugging in \eqref{eq:hardening} along with the relevant cosmographic relationships (see \secref{sec:app_eqs}) into \eqref{eq:num_num_dens} we have,
    %     \begin{equation}
    %         \frac{d^4 N}{dM \, dq \, dz \, d\ln\!\frst} =
    %             \frac{d^3 n_c}{dM \, dq \, dz} \, 4\pi c \, d_c^2 \lr{1+z} \, \thardf.
    %     \end{equation}


    \subsection{Semi-Analytic Modeling}

        Define the distribution function of sources as $F(M,q,a,z) = d^2 n(M,q,a,z) / dM dq$.  Here $M$ is the total mass of each systems, the mass-ratio is $q\equiv m_2/m_1 \leq 1$, $a$ is the binary separation, and $z$ is the redshift.
        % Define the distribution function of sources as $F(M,q,f_r,z) = d^2 n(M,q,f_r,z) / dM dq$.  Here $M$ is the total mass of each systems, the mass-ratio is $q\equiv m_2/m_1 \leq 1$, $f_r$ is the binary orbital frequency in the rest frame, and $z$ is the redshift.
        We can write the conservation equation for binaries as of function of redshift as,
        \begin{equation}
            \label{eq:conservation}
            \diffp{F}{z} +
                \diffp{}{M} \lrs{F \diffp{M}{z}} +
                \diffp{}{q} \lrs{F \diffp{q}{z}} +
                \diffp{}{a} \lrs{F \diffp{a}{z}} = S_{\!F}(M, q, a, z).
        \end{equation}
        Here $S_{\!F}$ is a source/sink function that can account for the creation or destruction of binaries.

        We consider the standard semi-analytic model (SAM) formalism of MBH binary populations \citep[e.g.~][]{Sesana+2008, Chen+2019}.  In this style of calculation, $F$ is determined in a region of parameter space that can be observed/estimated, and this is evolved to find the distribution in a different region of parameter space that is of interest.  In practice, the observed parameter space is galaxies and galaxy mergers, and the parameter space of interest is closely separated MBH binaries that could be GW detectable.  Thus we assume that all binary `formation' is encapsulated from binaries moving from one part of parameter space (i.e.~large separations and redshifts) to other parts of parameter space (i.e.~smaller separations and redshifts), and we set $S_{\!F} = 0$.

        We will express the distribution function as a product of a mass function, and a pair fraction:
        \begin{equation}
            \label{eq:dist_func}
            F(M,q,z) = \frac{\Phi(M, z)}{M \ln\!10} \cdot P(M,q,z),
        \end{equation}
        where the mass function, $\Phi(M, q, z) \equiv \diffp{n_g}{{\logten{M}}}$, is calculated based on the number density of galaxies ($n_g$).  We assume that there is a one-to-one mapping from galaxy mass to MBH mass, such that the galaxy mass-function can still be used to uniquely define the mass distribution of MBHs.  Typically the MBH--galaxy relation is given in terms of an `$M_\trt{BH}$--$M_\trt{bulge}$' relation \citep[e.g.][]{Kormendy+Ho-2013}, which is an observationally derived relation between the mass of MBHs and the mass of the stellar bulge component of their host galaxy.
        In \citet{Chen+2019}, the pair fraction is measured over some range of separations, and the separation-dependence is suppressed, i.e.~$P = \int_{a_0}^{a_1} P_a \, da$.

        From $\eqref{eq:conservation}$, we use the chain rule to mix time and redshift evolution, and assume that the mass-change of binaries is negligible, i.e.~$\diffp{m}{t} = 0$ and $\diffp{q}{t} = 0$, giving:
        \begin{equation}
            \diffp{F}{z} = - \diffp{t}{z} \diffp{}{a} \lrs{F \diffp{a}{t}}.
        \end{equation}

        The binary population is assumed to be changing only in separation and redshift, which are related by $\partial a / \partial z = (\partial a / \partial t) (\partial t / \partial z)$.  Because the overall number-density is conserved, we can take a finite step in separation and time/redshift, $a\rightarrow a'$ and $z\rightarrow z'$.  Here the time it takes for a binary to go from $a \rightarrow a'$ is $T(M,q,a,z|a')$, which leads to a redshift at the later time of $z' = z'(t + T)$.  So far, we have left the binary separation $a$ as implicit in the expression for F.  To obtain the standard expression \citep[e.g.][~Eq.~5]{Chen+2019}, we make the approximation that,
        \begin{equation}
            \diffp{}{a} \lrs{F(M,q,z) \diffp{a}{t}} \approx \frac{F}{T(M,q,a,z|a')}.
        \end{equation}
        Thus giving,
        \begin{equation}
            \label{eq:cont_eq_result}
            \diffp{F(M,q,a',z')}{{z'}} = \diffp{n}{{M}{q}{z'}} = - \diffp{t}{z} \frac{\Phi(M,z) \, P(M,q,z)}{T(M,q,a,z|a')}.
        \end{equation}

        Combining \eqref{eq:cont_eq_result} with \eqref{eq:num_num_dens}, we can finally write,
        \begin{equation}
            \label{eq:sam_final}
            \diffp{N}{{M}{q}{z}{\ln\!f_r}} = \frac{\Phi(M,z) \, P(M,q,z)}{T(M,q,a,z|a')} \, \thardf \, \diffp{V_c}{z}.
        \end{equation}

        From the SAM formalism, the number of binaries in each bin is calculated by `integrating' each finite element of \eqref{eq:sam_final}, i.e.,
        \begin{equation}
            \diffp{N_{ijk}(f_r)}{{\ln\!f_r}} \equiv \diffp{N(M_i, q_j, z_k, {f_r})}{{M}{q}{z}{\ln\!f_r}} \, \Delta M \,\Delta q \, \Delta z.
        \end{equation}
        The characteristic strain of the GWB can then be calculated

    \subsection{Finite Volume Populations: the Illustris simulations}

        \todo{Describe illustris models.}



\section{Implementation}
    \label{sec:imp}

    The \holodeck{} framework is written primarily in \python{} using a class-based structure for modularity and extensibility.

    \subsection{Semi-Analytic Models}
        \label{sec:imp_sam}

        The core of the semi-analytic modeling framework in \holodeck{} is implementing \eqref{eq:sam_final}.  There are classes for each of the key components: The galaxy stellar-mass function (GSMF, $\Phi$), the galaxy pair fraction \(F\), the merger timescale ($\tau$), and additional a scaling relation for mapping host galaxies via to MBH properties --- typically via an \mmbulge{} relation.  Each component class is implemented generically via basic API methods (e.g.~a GSMF that returns the number density of galaxies at a given stellar-mass and redshift), and can be subclassed to provide a particular implementation (e.g.~an analytic Schechter-function GSMF).  The number (or number density) of binaries is specified over a grid of parameters, typically in a four dimensional space of total mass ($M$), mass ratio ($q$), redshift ($z$), and binary rest-frame orbital frequency ($f_r$).

        The semi-analytic model formalism defines a continuous distribution of MBH binaries.  For many calculations, the discrete nature of MBHB sources is important, not only for examining individual continuous wave sources but also because of the important effects of discrete sources on the shape of the GWB due to the non-linear scaling of GW strain \citep{Sesana+2008}.  Additional, a component of the uncertainty in expected GW signals is due to `cosmic variance', which can be treated as Poisson noise when creating multiple samples (or `realizations') of our binary populations.  To handle discretization and resampling we use the \texttt{kalepy} package \citep{kalepy2021}, in particular, a method of outlier sampling to capture regions of parameter space where finite number effects are important.  The \holodeck{} semi-analytic models specify the number (or number-density) of binaries over a grid, i.e.~$N_{ijkl}$.  By definition, grid cells which have an expected total number of binaries $\gg 1$ will not produce appreciable finite-number effects, so instead they are given a weighting based on a Poisson distribution centered around their expectation value, i.e.~$\mathcal{P}(N_{ijkl})$.  Grid cells with expected numbers of order unity or less do produce important finite-number effects, and thus are directly resampled

        \subsubsection{Parameters}

            \begin{itemize}
                \item \textbf{GSMF}
                \begin{itemize}
                    \item \citep{Tomczak+2014} - single schechter fits, little variation over redshift\\
                    \item two
                \end{itemize}


                \item \textbf{GPF}
                \begin{itemize}
                    \item one \\
                    \item two
                \end{itemize}


            \end{itemize}



% =================================================================================================
% ====    Section 3 - Results
% =================================================================================================



\section{Results}
    \label{sec:res}

    % Fig -
    % \begin{figure*}
    %     \centering
    %     \includegraphics[width=1.5\columnwidth]{{{figs/det-nums_fedd-1.00_voff+3.00_dvel+2.00_tobs+0.70}}}
    %     \caption{\textbf{Distributions of parameters for all binaries (black) and those with a detectable kinematic signature in the secondary AGN (colors).}  Less than $10^{-7}$ of primaries are observable, which are not shown.  Secondary AGN with observable velocity offsets ($\voff$) are shown in blue assuming a sensitivity of \mbox{$\voffsens = 10^3 \kmps$}.  Secondaries with observable changing velocities ($\dvel$) are shown in orange (solid) for a sensitivity $\dvelsens = 10^2 \kmps$, and observing baseline of \mbox{$\tobs = 5\, \yr$}.  Systems where both offsets and changes are detectable (orange, dotted) are mostly a uniform subset of the $\dvel$ sample, and are almost indistinguishable in the 2D contour plots.  An optical flux cutoff of \mbox{$6\E{-14} \mathrm{erg/s/cm}^2$} (AB Mag $\approx 21$) is also imposed, where all AGN are assumed to be accreting at an Eddington fraction of $0.1$.}
    %     \label{fig:det_bins_num}
    % \end{figure*}



% =================================================================================================
% ====    Section 4 - Conclusions
% =================================================================================================



\section{Discussion \& Conclusions}
    \label{sec:disc}



% =================================================================================================
% ====    End Matter
% =================================================================================================


% ====    Acknowledgments



\section*{Acknowledgments}
	I am very thankful to.

    This research made use of \texttt{astropy}, a community-developed core Python package for Astronomy \citep{astropy2013}, in addition to \texttt{Scipy}~\citep{scipy}, \texttt{ipython}~\citep{ipython}, \texttt{jupyter}~notebook~\citep{jupyter}, \texttt{Numpy}~\citep{numpy2011} \& \texttt{SymPy}~\citep{sympy2017}.  All figures were generated using \texttt{matplotlib}~\citep{matplotlib2007}.  Kernel density estimation was performed using the \texttt{kalepy}{} package (\href{https://github.com/lzkelley/kalepy}{github.com/lzkelley/kalepy}) \citep{kalepy2021}.

    The Illustris data is available online at \href{https://www.illustris-project.org/}{www.illustris-project.org} \citep{Nelson+2015}, and Illustris-TNG data at \href{https://www.tng-project.org/}{www.tng-project.org} \citep{Nelson+2019}.


% ====    Bibliography

\let\oldUrl\url
\renewcommand{\url}[1]{\href{#1}{Link}}

\quad{}
\bibliographystyle{mnras}
\bibliography{src/refs}

\onecolumn
\clearpage


% ====    Appendices

\appendix



\section{Additional Equations}
    \label{sec:app_eqs}

    \subsection{Binaries and Gravitational Waves}

        A binary is described by its total mass $M$, mass ratio $q \equiv m_2 / m_1$ (s.t.~$q\leq1$), (proper) semi-major axis $a$, and rest-frame orbital frequency $f_p$.  The characteristic gravitational wave mass is the chirp mass, $\mchirp \equiv M \, q^{3/5} \, \lr[-6/5]{1+q}$.

        The time evolution of a binary's orbit due to the loss of energy and angular momentum through gravitational waves was calculated by \citep[][Eq.~5.6--5.8]{Peters1964}:
        \begin{subequations}
        \label{eq:gw_hard}
        \begin{align}
            \frac{da}{dt} = & -\frac{64 \, G^3}{5 \, c^5} \frac{M_1 \, M_2 \, M}{a^3} F(e), \\
            \frac{de}{dt} = & -\frac{304 \, G^3}{15 \, c^5} \frac{M_1 \, M_2 \, M}{a^4} \lr{e + \frac{121}{304}e^3},
        \end{align}
        \end{subequations}
        where the GW eccentricity function is,
        \begin{equation}
            \label{eq:ecc_func}
            F(e) \equiv \frac{1 + \frac{73}{24} e^2 + \frac{37}{96} e^4}{\left( 1 - e^2\right)^{7/2}}.
        \end{equation}
        We define the binary `hardening' timescale with respect to frequency, instead of separation, as:
        \begin{equation}
            \label{eq:hard_time}
            \thard \equiv dt/d\ln \frst = \frst / (d\frst/dt) = - \frac{2}{3} a / (da/dt).
        \end{equation}
        In general, the hardening rate can be driven by other mechanisms in addition to gravitational wave emission.

        GW power is emitted at integer harmonics of the orbital frequency, $\frstn = n \cdot \frstorb$, which is also redshifted to reach the observer-frame, $\fobsn = n \cdot \frstorb / (1+z)$.  The energy spectrum emitted in GWs is \citet[][Eq.~3.10]{enoki2007a},
        \begin{subequations}
        \begin{align}
            \label{eq:gw_energy_spectrum}
            \frac{d \egw}{d \frst} & = \int \frac{dt_r}{df_p} \, \sum_{n=1}^\infty \, \lgwn(f_p) \; \delta(\frst - n f_p) \; d f_p \\
                & = \sum_{n=1}^\infty \left[ \lgwn(f_p) \cdot \frac{\thard(f_p)}{n f_p} \right]_{f_p = \frst/n},
        \end{align}
        \end{subequations}
        where the power-radiated at each harmonic can be expressed as is \citep[][Eq.~2.2]{enoki2007a},
        \begin{align}
            \label{eq:lum_gw}
            \lgw & = \sum_n \lgwn = \lgwc \sum_n g(n,e) = \lgwc \cdot F(e), \\
            \lgwc & = \frac{32}{5 G c^5} \left(G\mchirp\right)^{10/3} \left( 2\pi \frstorb \right)^{10/3}.
        \end{align}
        Here, the GW frequency distribution function is,
        \begin{equation}
        \begin{split}
            \label{eq:freq_dist_func}
            g(n,e) \equiv \frac{n^4}{32} \Biggl(& \left[ J_{n-2}(ne) - 2eJ_{n-1}(ne) + \frac{2}{n} J_n (ne) + 2e J_{n+1} (ne) - J_{n+2} (ne) \vphantom{\frac{2}{n}}\right]^2 \\
            &+ \left(1-e^2\right) \Bigl[J_{n-2}(ne) - 2eJ_n(ne) + J_{n+2}(ne)\Bigr]^2 + \frac{4}{3n^2}\Bigl[J_n(ne)\Bigr]^2 \Biggr).
        \end{split}
        \end{equation}
        The GW spectral strain amplitude at a given harmonic can be related to the (rest-frame) luminosity as \citep[][Eq.~2.1]{Finn+Thorne-2000},
        \begin{align}
            \label{eq:strain_lum}
            \hsn^2 & = \frac{G}{c^3} \, \scale[2]{2}{n} \, \frac{\lgwn}{\lr[2]{2 \pi \, \frstorb} \, \distcom^2}, \\
                & = \frac{32}{5 c^8} \, \scale[2]{2}{n} \, \frac{\left(G\mchirp\right)^{10/3}}{\distcom^2} \lr[4/3]{2\pi\frstorb} \cdot g(n,e).
        \end{align}
        We always take the chirp mass to be an intrinsic (rest-frame) property of the binary, however an `observer-frame' chirp-mass can also be used.  For example, in the case of a circular binary, the spectral strain is,
        \begin{equation}
            \label{eq:gw_strain_amp_circ}
            \hscirc
                = \frac{8}{10^{1/2}} \frac{\left(G\mchirp\right)^{5/3}}{c^4 \, \distcom} \left(2 \pi \frstorb \right)^{2/3}
                = \frac{8}{10^{1/2}} \frac{\left(G\mchirp_o\right)^{5/3}}{c^4 \, \distlum} \left(2 \pi \fobsorb \right)^{2/3}.
        \end{equation}


    \subsection{Cosmology}
        Some useful cosmographic relations are reproduced here \citep{Hogg-1999}.  Recall that luminosity distance and comoving distance are related as $\distlum = (1+z) \distcom$.  The differential comoving volume of the Universe is,
        \begin{equation}
            \label{eq:comvol}
            \frac{d V_c}{dz} = 4\pi \, d_c^2 \, \frac{c}{H(z)},
        \end{equation}
        and time (or age of the universe) can be related to redshift through,
        \begin{align}
            \label{eq:redshift_time}
            \frac{dz}{dt} = \lr{1+z} \; H(z).
        \end{align}
        The Hubble parameter can be calculated from the matter-energy constituents of the universe as,
        \begin{align}
            H(z) & = H_0(z) \cdot \lrs[1/2]{\Omega_R \lr[4]{1+z} + \Omega_m \lr[3]{1+z} + \Omega_k \lr[2]{1+z} + \Omega_\Lambda}.
        \end{align}
        The $\Omega$ terms are the usual redshift-zero fractions of the critical density for radiation, non-relativistic matter, curvature, and dark energy.  We adopt the results of WMAP9 \citep{Hinshaw+2013} for our fiducial cosmology, assuming zero curvature: $H_0 = 69.33 \textrm{ km/s/Mpc}$, $\Omega_m = 0.2880$, $\Omega_\Lambda = 0.7120$.



\section{Additional Material}
    \label{sec:app}

    \noindent\begin{minipage}{\linewidth}

        % Fig -
        % \centering
        % \includegraphics[width=0.55\columnwidth]{{{figs/doan+2019_jitter}}}
        % \captionof{figure}{Distribution of velocity-offset variations (`jitters') from \citep{Doan+2019}, for the red and blue components of the observed BLs.  Measurements are shown as ticks at $y=0$, and KDE distributions are shown using a Scott's factor bandwidth, and Gaussian kernel.  The purple curve is the KDE distribution taking both red and blue data as independent.  The KDE $50\%$ (median) and $99\%$ (Gaussian standard deviation $\sigma \approx 2.3$) jitter for each component are shown with dashed vertical lines and corresponding labels.}
        % \label{fig:jitter_doan2019}
        %
        % \setlength{\tabcolsep}{8pt}
        % \begin{tabular}{p{1.5cm} c c c c c c c}
        %      & & & $5\%$ & $25\%$ & $50\%$ & $75\%$ & $95\%$ \\ [0.5ex]
        %     \hline
        %     \multirow{4}{*}{All}  & \multirow{4}{*}{-}
        %         & $M \; [M_\odot]$ & $3.4 \times10^{ 6 }$ & $7.2 \times10^{ 6 }$ & $1.8 \times10^{ 7 }$ & $6.0 \times10^{ 7 }$ & $4.1 \times10^{ 8 }$ \\
        %         & & q & $2.9 \times10^{ -2 }$ & $1.7 \times10^{ -1 }$ & $4.0 \times10^{ -1 }$ & $6.8 \times10^{ -1 }$ & $9.3 \times10^{ -1 }$ \\
        %         & & z & 0.24 & 0.41 & 0.53 & 0.62 & 0.69 \\
        %         & & $p \; [\mathrm{yr}]$ & $1.4 \times10^{ 2 }$ & $1.5 \times10^{ 3 }$ & $9.6 \times10^{ 3 }$ & $4.0 \times10^{ 4 }$ & $2.0 \times10^{ 5 }$ \\
        %         & & $a \; [\mathrm{pc}]$ & $2.1 \times10^{ -2 }$ & $1.3 \times10^{ -1 }$ & $4.9 \times10^{ -1 }$ & $1.3 \times10^{ 0 }$ & $4.1 \times10^{ 0 }$ \\
        %         & & $a \; [r_g]$ & $1.2 \times10^{ 4 }$ & $1.4 \times10^{ 5 }$ & $4.5 \times10^{ 5 }$ & $1.2 \times10^{ 6 }$ & $4.1 \times10^{ 6 }$ \\
        %     \hline
        %     \multirow{4}{1.5cm}{Secondary\newline Offset\newline($\voff$)} & \multirow{4}{*}{$0.49\%$}
        %         & $M \; [M_\odot]$ & $1.4 \times10^{ 8 }$ & $4.0 \times10^{ 8 }$ & $7.6 \times10^{ 8 }$ & $1.4 \times10^{ 9 }$ & $3.7 \times10^{ 9 }$ \\
        %         & & q & $9.6 \times10^{ -4 }$ & $4.2 \times10^{ -3 }$ & $1.2 \times10^{ -2 }$ & $3.6 \times10^{ -2 }$ & $1.5 \times10^{ -1 }$ \\
        %         & & z & 0.25 & 0.42 & 0.54 & 0.63 & 0.69 \\
        %         & & $p \; [\mathrm{yr}]$ & $8.3 \times10^{ 2 }$ & $1.5 \times10^{ 3 }$ & $2.3 \times10^{ 3 }$ & $3.7 \times10^{ 3 }$ & $9.0 \times10^{ 3 }$ \\
        %         & & $a \; [\mathrm{pc}]$ & 0.22 & 0.41 & 0.58 & 0.87 & 1.82 \\
        %         & & $a \; [r_g]$ & $5.0 \times10^{ 3 }$ & $1.1 \times10^{ 4 }$ & $1.8 \times10^{ 4 }$ & $2.9 \times10^{ 4 }$ & $4.9 \times10^{ 4 }$ \\
        % \end{tabular}
        % \captionof{table}{\textbf{Parameters of detectable binary systems with redshift $z < 0.7$.}  The indicated quantiles are given for total mass ($M$), mass ratio ($q$), separation ($a$), and orbital period ($p$).  In only four binaries ($2\E{-8}$ of systems) is the primary detectable, and those have parameters: \mbox{$M\approx 2$--$5\E{9} \, \msol$},
        % \hspace{0.5ex} \mbox{$q \approx 0.7$--$0.9$}, \hspace{0.5ex} \mbox{$z \approx 0.6$--$0.8$}, \hspace{0.5ex} \mbox{$a \approx 2.2$--$2.8 \, \pc$}.}
        % \label{tab:obs_pars}

    \end{minipage}

\twocolumn


\end{document}
